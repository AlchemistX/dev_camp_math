\chapter{집합과 함수}

본 단원에서는 집합과 그 표현, 그리고 집합의 구현에 대해서 다룹니다. 

\section{집합} 

\subsection{집합의 정의}

\begin{define}[집합]
특정 조건에 맞는 원소들의 모임. 임의의 한 원소가 그 모임에 속하는지를 알 수 있고, 그 모임에 속하는 임의의 두 원소가 다른가 같은가를 구별할 수 있는 명확한 표준이 있는 것을 이른다. \footnote{국립국어원의 정의 2번 참조.} \footnote{사실 이 정의만 국립국어원이 정의를 따르는데, 이는 집합 자체가 정의하기 매우 어렵기 때문입니다. 여기서는 간단하게 위 정의를 따르고 넘어갑니다.} 
\end{define}

\lstinputlisting[style=python, 
                firstnumber = 3, 
                linerange = {3-5, 77-79}, 
                caption={Set 구현 (PySet.py)}]{"../../src/pymath/PySet.py"}


\subsection{집합의 표기} 
% 조건제시법, 원소나열법 

\subsection{집합의 연산} 
% 합집합, 교집합, 여집합, 부분집합, 카테시안 곱 

\subsection{집합의 종류} 
% 집합의 크기에 따라 : 유한집합, 가산집합, 비가산집합 
% 집합의 순서에 따라 : ordered set, partially ordered set 

\subsection{집합의 예시 : 수 체계}
% 자연수(짝수 등), 유리수, 실수, 허수

\section{좌표계}
% 직교 좌표계 

\section{함수}

\subsection{함수의 정의} 
% just a python function, but we will introduce a wrapper! 


\subsection{함수의 종류} 
% 전사, 단사, 전단사

\subsection{합성함수와 역함수} 


\subsection{다양한 함수들} 
% 다항함수, 유리함수, 지수함수, 삼각함수, 로그함수 등 
