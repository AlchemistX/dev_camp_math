\chapter*{서문}

\section*{강의 소개}
본 강의는 프로그래머를 위한 수학 강의용 교재입니다. 학습자가 프로그래머이므로, 수학을 프로그래밍을 이용하여 배울 것입니다. 더 자세하게는, 수학에의 다양한 개념 - 집합, 함수, 식 등 - 을 파이썬으로 구현하는 것을 목표로 합니다.  

본 강의는 다음에 대한 내용들을 다룹니다. 

\begin{compactitem} 
\item 수학 
\begin{compactitem} 
\item 집합의 정의와 그 연산 
\item 함수의 정의와 표현, 식의 계산 
\item 행렬과 벡터의 정의, 연산 및 응용 
\item 미분, 적분의 정의와 그 응용 
\item 확률 및 통계에 대한 소개 
\end{compactitem}
\item 프로그래밍 
\begin{compactitem} 
\item 객체지향적 프로그램 연습 
\item generator, lambda expression 등 다양한 프로그래밍 기법 연습
\item 간단한 파싱과 인터프리터 작성 연습
\end{compactitem}
\end{compactitem}

\section*{교재의 구성} 

교재에서는 새로 나온 개념을 우선 수학적인 언어로 설명하고, 그에 해당하는 구현을 같이 소개합니다. 그리고 덧붙여 생소할 수 있는 프로그래밍 개념도 설명합니다.


\section*{사전지식}

이 강의를 듣기 위해서는 중학교 수준의 수학에 대한 지식과 더불어, 재귀나 객체지향에 대한 기본적인 이해가 필요합니다. 만약 대학교 공업수학 정도의 수학에 익숙하면서 동시에 프로그래밍에 대한 지식이 있다면 이 강의를 들을 필요가 없습니다. 


\section*{강의 자료}

본 교재는 패스트캠퍼스의 Coding the Mathematics 강의\footnote{\href{http://www.fastcampus.co.kr/dev_camp_math/}{강의 웹페이지}}를 위해서 제작되었습니다. 본 강의에서 구현할 코드들과 그 스켈레톤은 깃헙\footnote{\href{https://github.com/principia12/dev_camp_math}{강의자료 깃헙 repo}}에서 찾아볼 수 있습니다. 제공된 소스 코드는 파이썬 3.5로 윈도우 환경에서 테스트되었습니다. 



